\chapter{Introdução}
\label{chap:Introducao}

Com o desenvolvimento do uso de grafos para modelar sistemas, muitos pesquisadores tem tido interesse com estudos sobre grafos \cite{joachim}.
Os objetivos desses pesquisadores são variados, que podem ser desde classificação do grafo \cite{fbclass}, à predição \cite{dota} e para uma compreensão melhor do mesmo \cite{rosvall2008maps}. 

Considerando o alvo da modelação das pesquisas que vão desde redes sociais \cite{fbclass}, a estruturas químicas \cite{rosvall2008maps} e jogos \cite{dota}, será destacado os jogos, pois ainda no meio acadêmico são novidades quando se trata de modelação em grafos.
Jogos que tem conquistado a atenção e o tempo de muitas pessoas, movimentando mais receita do que a industria de filmes Hollywood\footnote{http://www.webnoticias.fic.ufg.br/n/68881-industria-de-games-supera-o-faturamento-de-hollywood}.

Com a popularização do \textit{League of Legends}, um jogo da categoria de MOBA, e a crescente premiação\footnote{http://www.espn.com.br/noticia/736116\_premiacao-do-mundial-de-league-of-legends-ultrapassa-us-4-milhoes} dos campeonatos do \textit{League of Legends}\footnote{https://play.br.leagueoflegends.com/pt\_BR}, jovens e adultos tem se tornado \textit{pro players} e dedicado sua vida ao mundo dos jogos\footnote{https://oglobo.globo.com/esportes/de-nerds-ciberatletas-crescimento-exponencial-do-sports-21233721}. 
Esse fato viabiliza o uso de ferramentas para ajuda-los na analise dos jogos, e de estratégias. 
E devido a popularidade do jogo e a facilidade de obtenção dos dados devido a API do \textit{League of Legends}, ele é escolhido como foco do trabalho.

Nesse contexto, o presente trabalho propõe a modelagem do grafo das partidas do jogo \textit{League of Legends}, a sua visualização, a classificação do mesmo e a predição de equipes vencedoras utilizando MLP, no formato de um \textit{webservice}.

\section{Contextualização}
Na matriz curricular do curso de Engenharia de Computação do Instituto Federal de Minas Gerais - \textit{campus} Bambuí, existem várias disciplinas que foram utilizadas no presente trabalho, como programação orientada a objetos, banco de dados, programação web dentre outras e com o estudo mais aprofundado sobre grafos, foi possível desenvolver o \textit{webservice} pois a API do LOL é fácil de ser usada e facilita a obtenção dos dados.


Tendo como proposta o presente trabalho o desenvolvimento de um \textit{webservice} que é possível predizer uma equipe vencedora pela escolha do herói dos jogadores, e também mostrar a sinergia entre os campeões e o seus \textit{counter picks}. Também será feito a classificação da modelação do grafo obtido dos dados do jogo, para fins acadêmicos.

\section{Objetivos}
Os objetivos desse trabalho podem ser divididos em objetivos gerais e específicos, que podem ser encontrados nas próximas subseções, junto também com os resultados esperados.
\subsection{Objetivos Gerais}
Criar um modelo computacional capaz de exibir resumos da rede complexa gerada dos dados armazenados e predizer um vencedor no jogo \textit{League of Legends} baseando-se num histórico de partidas usando MLP.

\subsection{Objetivos Específicos e Resultados Esperados}

\begin{enumerate}
\item Construir um \textit{webservice} para visualização;
\item Calcular deterministicamente quantas vezes cada herói ganhou contra/com outro herói;
\item Testar o resultado obtido em outra amostra de dados;
\item Documentar os resultados.

\end{enumerate}

\section{Justificativa}
Devido ao grande crescimento e a popularização do jogo \textit{League of Legends}, tem surgido muitos sites de análises do mesmo que vem auxiliando tanto os jogadores esporádicos, tanto quanto os \textit{pro players}.
Este trabalho propõe desenvolver um modelo computacional para que usuários, \textit{coach} ou analistas possam ter uma ferramenta para obtenção de informações ou para predição do resultado de um jogo.

A utilização da rede complexa modelar os dados, foi escolhida porque assim é possível representar quantificadamente as conexões de cada herói, sendo conexões aliadas ou adversárias.

Assim, este trabalho se justifica principalmente pela análise da rede complexa obtida, pela visualização gráfica e pela predição de partidas para que essas informações adquiridas se tornem acessíveis. E também para execução desse trabalho é necessário o uso de várias áreas do conhecimento estudadas no curso de Engenharia de Computação, principalmente computação.