\chapter{Introdução}
\label{chap:Introducao}

Quando descrevemos uma situação utilizando pontos e algumas ligações entre esses pontos, estamos modelando este cenário em forma de grafos. E quando algo está modelado em forma de grafos, podemos usar tecnologias que são aplicáveis em grafos.



%Quando descrevemos uma situação utilizando pontos e ligações entre algum desses pontos, como o exemplo os pontos sendo pessoas, e as ligações entre as pessoas sendo as amizades feitas, a abstração matemática desse tipo dá lugar ao conceito de grafo \cite{Lucchesi1979}. O termo redes complexas refere-se a um grafo que é composto por um conjunto de pontos ( vértices ), com arestas ligando algum desses pontos \cite{Albert2002}.

%As redes complexas são capazes de descrever sistemas utilizando a teoria de grafos, como por exemplo, uma rede complexa de roteadores e suas conexões físicas ou sem fio \cite{Albert2002}, ou cidades e a suas distâncias sendo as ligações.

%Também é possível representar um jogo de futebol, onde vamos ter dois grupos de pontos, que são os jogadores de cada time, e as ligações entre os do mesmo time podem ser por exemplo, a quantas vezes jogaram juntos e quantas ganharam, e as ligações entre os adversários, quantas vezes se enfrentaram, quantas ganharam e quantas perderam.

%Levando essa forma de abstração da rede complexa para o jogo chamado \textit{League of Legends}, onde existem duas equipes adversárias e cinco jogadores em cada equipe. Os pontos seriam os heróis escolhidos por cada jogador, e as arestas as afinidades entre os heróis.

%Depois de formado a rede complexa, será possível a visualização dessas informações em um \textit{webservice}, e será possível tentar predizer o resultado de um jogo com base nas escolhas feitas no início do jogo.


\section{Objetivos}

\subsection{Objetivos Gerais}
Criar um modelo computacional capaz de visualizar e predizer a possibilidade de vitórias e derrotas no jogo \textit{League of Legends} baseando-se num histórico de partidas usando redes complexas.

\subsection{Objetivos Específicos e Resultados Esperados}

\begin{enumerate}
\item Construir um \textit{webservice} para visualização;
\item Calcular deterministicamente quantas vezes cada herói ganhou contra/com outro herói;
\item Testar o resultado obtido em outra amostra de dados;
\item Documentar os resultados.

\end{enumerate}

\section{Justificativa}
Devido a crescente popularidade dos videogames no mundo e seus campeonatos mundiais valendo milhões de dólares, a criação deste modelo computacional capaz de visualizar e predizer a possibilidade de vitórias e derrotas, [continua...]