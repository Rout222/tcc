\chapter{Introdução}
\label{chap:Introducao}

Atualmente, muitos jogos tem conquistado a atenção e o tempo de muitas pessoas. Com a nova categoria chamada \textit{e-sport} muitas pessoas, têm dedicado o seu tempo aos jogos como forma de trabalho, já que a premiação só vem aumentando com o passar dos anos.

Premiação essa que, segundo a \citet{espn}


E como todo jogo que existem equipes adversárias, no \textit{League of Legends}, existem vencedores e perdedores. E este trabalho vem com a intenção de prever os vencedores de uma partida deste jogo utilizando grafos.

Quando descrevemos uma situação utilizando pontos e algumas ligações entre esses pontos, estamos modelando este cenário em forma de grafos. E quando algo está modelado em forma de grafos, podemos usar tecnologias que são aplicáveis em grafos.

\section{Objetivos}
Os objetivos desse trabalho, podem ser divididos em objetivos gerais, e específicos. Que podem ser encontrados nas próximas subseções, junto também com o resultados esperados.
\subsection{Objetivos Gerais}
Criar um modelo computacional capaz de visualizar e predizer a possibilidade de vitórias e derrotas no jogo \textit{League of Legends} baseando-se num histórico de partidas usando redes complexas.

\subsection{Objetivos Específicos e Resultados Esperados}

\begin{enumerate}
\item Construir um \textit{webservice} para visualização;
\item Calcular deterministicamente quantas vezes cada herói ganhou contra/com outro herói;
\item Testar o resultado obtido em outra amostra de dados;
\item Documentar os resultados.

\end{enumerate}

\section{Justificativa}
Devido a crescente popularidade dos videogames no mundo e seus campeonatos mundiais valendo milhões de dólares, a criação deste modelo computacional capaz de visualizar e predizer a possibilidade de vitórias e derrotas, [continua...]