\chapter{Introdução}
\label{chap:Introducao}

Muitos pesquisadores tem tido interesse em estudar sobre grafos, pois estes estão sendo desenvolvidos para modelizar sistemas \cite{joachim}, como, por exemplo, rede de energia de Nova York e a rede alimentar de Little Rock Lake \cite{strogatz}.
Os objetivos desses pesquisadores são variados, que podem ser desde classificação de grafos \cite{fbclass}, à predição \cite{dota} e para uma compreensão melhor do mesmo \cite{rosvall2008maps}. 

Alguns exemplos de modelação das pesquisas são redes sociais \cite{fbclass}, estruturas químicas \cite{rosvall2008maps} e jogos \cite{dota}. Em nosso trabalho, trabalharemos na de jogos, pois ainda no meio acadêmico são novidades quando se trata de modelação em grafos.
Jogos que tem conquistado a atenção e o tempo de muitas pessoas, movimentando mais receita do que a industria de filmes Hollywood\footnote{http://www.webnoticias.fic.ufg.br/n/68881-industria-de-games-supera-o-faturamento-de-hollywood}.

Com a popularização do \textit{\textbf{L}eague \textbf{o}f \textbf{L}egends}\footnote{https://play.br.leagueoflegends.com/pt\_BR}, também conhecido como LOL, um jogo da categoria de MOBA (\textbf{M}ultiplayer \textbf{O}nline \textbf{B}attle \textbf{A}rena, ou, pela tradução, "Arena de batalha online de multijogadores"), e sua crescente premiação em campeonatos\footnote{http://www.espn.com.br/noticia/736116\_premiacao-do-mundial-de-league-of-legends-ultrapassa-us-4-milhoes}, jovens e adultos tem se tornado \textit{pro players} (de tradução: jogadores profissionais) e dedicado sua vida ao mundo dos jogos\footnote{https://oglobo.globo.com/esportes/de-nerds-ciberatletas-crescimento-exponencial-do-sports-21233721}. 
Esse fato viabiliza o uso de ferramentas para ajudá-los na análise dos jogos e de estratégias. 
Devido a popularidade do jogo e a facilidade de obtenção dos dados devido a API ( \textbf{A}pplication \textbf{P}rogramming \textbf{I}nterface ou, da tradução, Interface de Programação de Aplicações) do LOL, ele é escolhido como foco do trabalho.

Nesse contexto, o presente trabalho propõe a modelagem do grafo das partidas do jogo \textit{League of Legends}, a sua visualização, a classificação do mesmo e a predição de equipes vencedoras utilizando MLP (Multilayer Perceptron, ou, da tradução, Perceptron Multicamadas), no formato de um \textit{webservice} (da tradução, serviço da internet).

\section{Contextualização}
Na matriz curricular do curso de Engenharia de Computação do Instituto Federal de Minas Gerais - \textit{campus} Bambuí, existem várias disciplinas que foram utilizadas no presente trabalho (como Programação Orientada a Objetos, Banco de Dados, Programação Web dentre outras) e com o estudo mais aprofundado sobre grafos, foi possível desenvolver o \textit{webservice} pois a API do LOL é fácil de ser usada e facilita a obtenção dos dados.


A proposta deste trabalho será o desenvolvimento de um \textit{webservice} que indicará uma possível equipe vencedora, de acordo pela escolha dos campeões (personagens jogáveis do jogo, em que cada um, individualmente, tem atributos e habilidades diferentes) dos jogadores, e, também, mostrar a sinergia entre os campeões e o seus \textit{counter picks} (da tradução, "contra-escolha", que seria uma expressão usada nos jogos de MOBA para indicar quando um campeão específico tem, por sí só, uma vantagem sobre outro campeão, também específico). Também será feito a classificação da modelação do grafo obtido dos dados do jogo para fins acadêmicos.

\section{Objetivos}
\subsection{Objetivos Gerais}
Criar um modelo computacional capaz de exibir resumos da rede complexa gerada dos dados armazenados e predizer um vencedor no jogo \textit{League of Legends} baseando-se num histórico de partidas usando MLP.

\subsection{Objetivos Específicos e Resultados Esperados}

\begin{enumerate}
\item Construir um \textit{webservice} para visualização;
\item Calcular deterministicamente quantas vezes cada herói ganhou contra/com outro herói;
\item Classificar o grafo;
\item Testar o resultado obtido em outra amostra de dados;
\item Documentar os resultados.

\end{enumerate}

\section{Justificativa}
Devido ao grande crescimento e a popularização do jogo \textit{League of Legends}, tem surgido muitos sites de análises do mesmo que vem auxiliando tanto os jogadores esporádicos, tanto quanto os \textit{pro players}.
Este trabalho propõe desenvolver um modelo computacional para que usuários, \textit{coach}, ou analistas possam ter uma ferramenta para obtenção de informações ou para predição do resultado de um jogo.

A utilização da rede complexa para modelar os dados, foi escolhida, pois ela possibilita representar quantificadamente as conexões de cada herói, sendo conexões aliadas ou adversárias.

Assim, este trabalho se justifica principalmente pela análise da rede complexa obtida, pela visualização gráfica e pela predição de partidas para que estas informações adquiridas se tornem acessíveis. 
Para execução desse trabalho é necessário o uso de várias áreas do conhecimento estudadas no curso de Engenharia de Computação, principalmente computação.