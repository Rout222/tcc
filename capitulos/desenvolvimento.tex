\chapter{Metodologia}
Este capítulo apresenta as ferramentas e algoritmos utilizados neste projeto tanto no processo de extração do conhecimento, quanto na visualização dos dados.
A primeira seção irá explicar como os dados estão disponíveis e de que informações eles carregam.  A segunda fala sobre a linguagem de programação utilizada para conseguir os dados, verificar se os dados estão completos, e armazena-los em um banco de dados. A terceira clarifica o que é um banco de dados, e porque foi utilizado.

\section{API do \textit{league of legends}}

A \textit{Riot Games}, desenvolvedora e dono do jogo \textit{League of Legends}, fornece uma Interface de Programação de Aplicativos ( do inglês \textit{Application Programming Interface} ) ou chamado de API para que terceiros consigam acessar dados sobre os jogos.

A Riot games disponibiliza o acesso as informações de partir à patir da URL\footnote{http://developer.riotgames.com} de modo que é gerado uma chave válida por 1 ano para projetos cadastrados.
O acesso à essa API é por URL utilizando uma função disponível pela \textit{RIOT Games}. Neste trabalho, usaremos apenas a função \textit{MATCH-V3}, que é uma função que retorna os dados de uma partida já terminada.

Nas próximas subseções serão abordados o formato no qual a API disponibiliza os dados, e quais informações ela retorna com o uso do \textit{MATCH-V3} e quais serão usados no projeto.

\subsection{MATCH-V3}

A função \textit{MATCH-V3} é disponibilizada publicamente pela \textit{RIOT Games} e que retorna um conjunto de informações no formato JSON sobre a partida passada por parâmetro, quando essa partida existe no banco de dados da \textit{RIOT Games}. A Figura \ref{fig:match-v3} resume um exemplo de acesso utilizando a função acessando a URL\footnote{https://br1.api.riotgames.com/lol/match/v3/matches/1381102031?api\_key=minhachave} acessada em 21 de março de 2018, sendo que \(minhachave\) tem que ser substituída por uma chave privada válida.
\begin{figure}[H]
	\caption{Exemplo de retorno do uso da função \textit{MATCH-V3}. Sendo as informações divididas em informações da partida, informações dos times, e informações dos participantes.}
	\begin{center}
		\includegraphics[width=9cm]{imagens/match-v3.jpg}
	\end{center}
	\small{Fonte: Autor.}
	\label{fig:match-v3}
\end{figure}

De todas essas informações retornadas pela função, as informações armazenadas para o projeto por participante são:

\begin{enumerate}
\item \textit{gameId}. Identificador único da partida;
\item \textit{kills}, \textit{deaths} e \textit{assists}. Informações que falam, respectivamente, quantas vezes esse jogador matou, morreu e participou na morte de outrem;
\item \textit{win}. Se ele ganhou;
\item \textit{championId}. Qual campeão ele escolheu;
\item \textit{lane}. Em qual posição ele tava jogando;
\item \textit{platformId}. Em qual servidor ele jogava;
\item \textit{queueId}. E qual o tipo de partida.
\end{enumerate}

Sabendo qual informação vamos armazenar, ficou escolhido a linguagem \textit{Python} para ser desenvolvido o algoritmo no qual obterá as informações de forma não manual e o sistema de gerenciamento de banco de dados MySQL. Na próxima seção será apresentado sobre banco de dados e na seção \ref{section:get} será esclarecido como foram obtidos o \textit{dataset} das partidas. 


\section{Banco de Dados}

\begin{figure}[H]
	\caption{Esquema do Banco de Dados.}
	\begin{center}
		\includegraphics[width=17cm]{imagens/esquema.png}
	\end{center}
	\small{Fonte: Autor.}
	\label{fig:novafigura}
\end{figure}


\section{Conseguindo os dados}
\label{section:get}
\textit{Python} é uma linguagem de programação dentre muitas, assim como existem várias linguagens faladas e escritas por nós humanos, no mundo da computação, existem várias linguagens de programação, onde geralmente cada uma se destaca em pelo menos um quesito. 

(pegar algum autor para respaldar, e pegar algum texto dele falando sobre vantagens do \textit{Python}, que é interpretado e continua.)

Entendendo isso, foi criado um algoritmo para a aquisição dos dados de partidas pela API do \textit{League of Legends}, que usava a biblioteca \textit{requests} do \textit{Python} para conseguir acessar a API por URL's, e a biblioteca json também do \textit{Python} para transformar o Json retornado pela requisição da URL em um objeto em \textit{Python}.

O aplicativo verifica se a partida requisitada existe e se os dados necessários estão completo, salvando no banco de dados apenas as partidas válidas. O pseudo-codigo se encontra no Algoritmo \ref{alg:main-aqui}.

\
 %Código


\begin{algorithm}[H]
   \SetAlgoLined
     \Inicio{

		i = 1000000000;\tcp*[f]{Esta é a \textit{id} de uma partida aleatória na ultima versão do jogo}
            
		chave = minha chave privada de acesso;
            
        \uSe{Existe arquivo com ultimo partida lido}{
          	i = ultima partida lida;
        }
        \Repita{Até ser pausado}{    
            pedido = requisita URL Da API(i, chave);

            \Se((\tcp*[f]{Se a partida existe}){pedido.status == 200 }{
            	 
            	JSON = carrega JSON Do Pedido(pedido);
                
                \uSe{JSON é válido}{
                	Armazena os dados em um banco de dados;
                }
                
                i = i + 1;

            }
		}
	                
	Salva i no arquivo;
    }
   \label{alg:main-aqui}
   \caption{\textsc{Aquisição dos dados das partidas}}
 \end{algorithm}

\

Onde com esse algoritmo foram armazenadas \numpartidas\ jogos válidos diferentes, e foi decidido usar apenas as partidas ranqueadas 5 contra 5, já que estas são as partidas responsáveis pela classificação dos jogadores dentro dos jogo, e as partidas não ranqueadas são consideradas como amistosas.
Com esse escopo, o número de partidas usados foram ao todo \partidasrankeds\ partidas ranqueadas diferentes, na tabela \ref{tab:subset-lol} é possível ver uma pequena fatia dos dados salvos.


\begin{table}[H]
\centering
\caption{Exemplo de \textit{subset} salvo no banco de dados}
\label{tab:subset-lol}
\resizebox{\textwidth}{!}{%
\begin{tabular}{cccccccccc}
match\_id  & kills & deaths & assists & win & champ\_id & lane   & player\_id & platform & type \\
1282000002 & 11    & 10     & 7       & 0   & 67        & BOTTOM & 16112724   & BR1      & 420  \\
1282000002 & 2     & 8      & 15      & 0   & 412       & BOTTOM & 18604874   & BR1      & 420  \\
1282000002 & 7     & 4      & 7       & 0   & 34        & MIDDLE & 19281809   & BR1      & 420  \\
1282000002 & 7     & 6      & 4       & 0   & 5         & JUNGLE & 21304223   & BR1      & 420  \\
1282000002 & 5     & 7      & 12      & 0   & 98        & TOP    & 651014     & BR1      & 420  \\
1282000002 & 0     & 6      & 23      & 1   & 44        & BOTTOM & 18592234   & BR1      & 420  \\
1282000002 & 19    & 5      & 5       & 1   & 222       & BOTTOM & 7170345    & BR1      & 420 
\end{tabular}%
}
\small{Fonte: Autor.}
\end{table}

Na etapa seguinte são contabilizados quantas vezes cada campeão jogou com outro campeão, seja no mesmo time ou no time adversário, calculado quantas vitória contra, e com o outro herói.
E com as informações já processados, foi montado um \textit{web service}, que sera explicado na secão \ref{chap:web}, para uma melhor visualização dos dados obtidos utilizando uma biblioteca chamada D3.js, que será explicado na próxima seção.

\section{D3.js}
A biblioteca D3.js, é uma biblioteca em JavaScript, especializada em dar vida aos dados. Segundo \citet{d3cook} "Em certo sentido, o D3 é uma biblioteca JavaScript especializada que permite criar incríveis
visualizações de dados usando uma abordagem mais simples (baseada em dados), aproveitando os padrões da Web existentes", e a organização oficial diz :


\begin{quote}
O D3.js é uma biblioteca em JavaScript para manipulação de documentos por dados. D3 te ajuda a trazer vida para os dados utilizando HTML, SVG e CSS. A enfase da biblioteca D3 nos padrões da \textit{web} dá-lhe todos os recursos dos navegadores modernos sem te prender a um \textit{framework} propietário, combinando poderosos componentes de visualização e uma aproximacão orientada por dados da manipulação do DOM.
\cite[tradução nossa]{d3js}
\end{quote}


Com essa biblioteca, é possível dar vida à informações e dados como por exemplo a Figura \ref{fig:d3}, duas das diversas maneiras de se exibir informações com o D3.js. E com o D3 foi gerado o grafo do \textit{web service} que sera melhor explicado na seção seguinte.

\begin{figure}[H]
	\caption{Exemplo de visualização de dados utilizando o D3.js.}
	\begin{center}
		\includegraphics[width=7.5cm]{imagens/d3_1.png}%
		\hspace{1cm}% add some horizontal spacing
		\includegraphics[width=7.5cm]{imagens/d3_2.PNG}%
	\end{center}
	\small{Fonte: \cite{d3js}.}
	\label{fig:d3}
\end{figure}


\section{\textit{web service}}
\label{chap:web}

O \textit{web service}, foi criado utilizando o \textit{framework} Flask, que como \citet[tradução nossa]{flask} diz "Flask é um micro \textit{framework} para Python baseado em Werkzeug, Jinja 2 e em boas intenções.".

Com essa serviço, será possível um relatório geral, onde ele consegue ver quais campeões são melhores contra e com quais, consegue filtrar quais campeões podem participar da pesquisa, e quais não podem. 

Nas figuras \ref{fig:web_service_relatorio} é possível ver um exemplo de uso do serviço para uma visão geral das informações. E na Figura \ref{fig:web_service_predict} é possível ver o uso da ferramenta para predição de uma batalha. Na seção \ref{chap:pred} será melhor abordado a maneira que essa predição é feita.



\begin{figure}[H]
	
	\centering
	\caption{Exemplo de uso do \textit{web service} para visão geral dos dados.}
	\subfloat[Filtro partida para campeões da mesma equipe]{
		\includegraphics[width=0.4\textwidth]{imagens/web_1.JPG}
		}
	\subfloat[Filtro para numero de partidas para adversários]{
		\includegraphics[width=0.4\textwidth]{imagens/web_2.JPG}
		}
	\qquad
	\subfloat[Filtro para escolher quais campeões podem participar (\textit{picks}), e quais nao podem (\textit{bans})]{
		\includegraphics[width=0.4\textwidth]{imagens/web_3.JPG}
	}
	\subfloat[Exemplo de grafo exibido pelo \textit{web service}, sendo as setas vermelhas mostrando quem é bom contra quem, e as linhas continuas quem é bom com quem, e o tamanho do circulo é a frequência daquela ligação, e a cor a força da ligação]{
		\includegraphics[width=0.4\textwidth]{imagens/web_4.JPG}
	}

	\small{Fonte: Autor.}
	\label{fig:web_service_relatorio}
\end{figure}

\begin{figure}[H]
	
	\centering
	\caption{Exemplo de uso do \textit{web service} predição da partida.}
	\subfloat[Esperando ser feito]{
		\rule{7cm}{7cm}
	}
	\subfloat[Esperando ser feito]{
		\rule{7cm}{7cm}
	}
	\qquad
	\subfloat[Esperando ser feito]{
		\rule{7cm}{7cm}
	}
	\subfloat[Esperando ser feito]{
		\rule{7cm}{7cm}
	}


	\small{Fonte: Autor.}
	\label{fig:web_service_predict}
\end{figure}

\section{Predição dos dados}
\label{chap:pred}
Como ainda não conseguimos dados suficiente, não chegamos até aqui.