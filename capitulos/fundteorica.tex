\chapter{Fundamentação Teórica}
\label{chap:fundteorica}
Nessa parte será apresentados conceitos para melhor entendimento deste trabalho e da modelagem utilizada. 
A próxima seção elucida o que são grafos e depois é explicado a variação de grafo utilizado para modelar o problema proposto. Logo em seguida, é esclarecido o ambiente que será abstraído, explicando algumas regras básicas do jogo.
\section{Grafos}
\label{chap:grafos}

Quando descrevemos uma situação utilizando pontos e ligações entre algum desses pontos, a abstração matemática desse tipo dá lugar ao conceito de grafo \cite{Lucchesi1979}.
Um grafo \(G(V, E)\) pode ser definido como um conjunto de vértices \(V\) e seu conjunto de arestas \(E\) sendo que cada elemento \(e \in E\), associa dois vértices do conjunto \(V\). Assim, se \((u,v) \in E\), então os vértices \(u\) e \(v\) são conectados pela aresta \(e\). Referindo-se a eles como vértices vizinhos ou adjacentes \cite{grafosucinto, Viana2007}.

Os grafos também podem ter suas arestas classificadas como binárias e ponderados, sendo os binários suas arestas representadas apenas por 1 ( se existem ) ou 0 ( se não existem ) e os ponderadas quando existe um conjunto \(W\) onde informa a intensidade da interação de uma aresta entre dois vértices  \cite{Viana2007, cormen6}.
Na Figura \ref{fig:grafo-exemplobinario} temos um grafo binário onde se \((u,v) \in E\) então ela está representada no grafo, já na Figura \ref{fig:grafo-exemploponderado} além da existência da conexão entre os vértices, está representado o conjunto \(W\) informando o peso daquela aresta \(e\).

\begin{figure}[!h] \centering
	\centering
	\caption{Exemplo de grafos.}
	\subfloat[Grafo binário com quatro vértices e quatro arestas.]{
		\begin{tikzpicture}
		\begin{scope}[xshift=4cm]
		\node[main node] (1) {$1$};
		\node[main node] (2) [right = 2cm  of 1] {$2$};
		\node[main node] (3) [below = 2cm  of 1] {$3$};
		\node[main node] (4) [right = 2cm  of 3] {$4$};
		
		\path[draw,thick]
		(1) edge node {} (2)
		(1) edge node {} (4)
		(4) edge node {} (2)
		(4) edge node {} (3)
		;
		\end{scope}
		\end{tikzpicture}
		\label{fig:grafo-exemplobinario}
	}
	\subfloat[Grafo não-binário com quatro vértices e três arestas.]{
		\begin{tikzpicture}
		\begin{scope}[xshift=4cm]
		\node[main node] (1) {$1$};
		\node[main node] (2) [right = 2cm  of 1] {$2$};
		\node[main node] (3) [below = 2cm  of 1] {$3$};
		\node[main node] (4) [right = 2cm  of 3] {$4$};
		
		\path[draw,thick]
		(1) edge node {2} (2)
		(1) edge node {1} (3)
		(2) edge node {1} (4)
		;
		\end{scope}
		\end{tikzpicture}
		\label{fig:grafo-exemploponderado}
	}
	\
	\small{\leftline{Fonte: Autor.}}
	\label{fig:grafo-exemplo}
\end{figure}

\citet{grafosucinto} afirma sobre a vizinhança de um grafo que "O conjunto de vértices \(X\) de um grafo \(G\) é o conjunto de todos os vértices que tem algum vizinho em \(X\)", e diz que esse conjunto de vértices pode ser chamado de \(\Gamma_G(X)\). Exemplificando, o conjunto \(\Gamma_G(1)\) da Figura \ref{fig:grafo-exemplobinario} são os vértices \(2\) e \(4\) e a aresta \( (2,4) \) sendo representados na Figura \ref{fig:grafo-vizinhanca}.

\begin{figure}[H] \centering
	\centering
	\caption{Exemplo de vizinhança de um vértice de um grafo.}
		\begin{tikzpicture}
		\begin{scope}[xshift=4cm]
		\node[targetnode] (1) {$1$};
 		\node[main node] (2) [right = 2cm  of 1] {$2$};
		\node[offnode] (3) [below = 2cm  of 1] {$3$};
		\node[main node] (4) [right = 2cm  of 3] {$4$};

		\path[draw, style={offline}]
		(1) edge node {} (2)
		(1) edge node {} (4)
		(3) edge node {} (4)
		;
		\path[draw,thick]
		(4) edge node {} (2)
		;
		\end{scope}
		\end{tikzpicture}
	\small{\leftline{Fonte: Autor.}}
	\label{fig:grafo-vizinhanca}
\end{figure}

\section{Propriedade topológicas}

O grau de um vértice \(v\) é definido sendo a quantidade de arestas que chegam no vértice \(v\), sendo denotado por \(g(v)\). Continuando ainda no exemplo da Figura \ref{fig:grafo-exemplobinario} o \(g(1)\) é 2 pois apenas as arestas \((1,2)\) e \((1,4)\) incidem no vértice 1 \cite{grafosucinto}.

O grau máximo de um grafo \(G\) é o grau do vértice que tem o maior grau presente nesse grafo \(V(G)\), ou seja \(\Delta(G) = max\{g(v) : v \in V(G)\}\). E o grau mínimo de um grafo \(G\) é o grau do vértice com menor grau, \(\delta(G) = min\{g(v) : v \in V(G)\}\). E ainda, um grafo é regular se \(\delta(G) = \Delta(G)\) e é \(k\)-regular se \(\delta(G) = \Delta(G) = k\) \cite{grafosucinto}.


Quando modelamos um sistema real em forma de grafos vamos ver na seção seguinte que esse grafo recebe um nome especial, sendo chamado de rede complexa.

\section{Redes Complexas}
\label{chap:redecomplexa}
Ainda sobre grafos \citet{Viana2007} afirma que é utilizado o termo redes complexas quando um grafo representa um sistema físico real, então um grafo do jogo LOL pode ser considerado como uma rede complexa.

Para que seja possível classificar o nosso modelo adequadamente, serão apresentados três modelos que se destacam entre os pesquisadores segundo \citet{albert2002statistical}: As redes \textit{small-worlds}, as redes livres de escala e as redes aleatórias. E também será explicado sobre o coeficiente de aglomeração para melhor caracterização do modelo adquirido.

\subsection{\textit{Redes Small-Worlds}}
 As redes \textit{small-worlds} são um modelo de rede complexas e este tipo de rede complexa representa uma alternativa ao modelo aleatório, assumindo como hipótese que redes complexas do mundo real presente no mundo animal, biológico e químico não são completamente aleatórias \cite{watts1998collective, lopes2011redes}. 
 
 Tendo um grafo regular com $v$ vértices e $k$ arestas ligando os vizinhos mais próximos, depois de formado o grafo cada aresta tem uma probabilidade $p$ de se reconectar com outro vértice aleatório. Permitindo que a geração do grafo seja controlado para uma rede regular com $p \approx 0$ ou uma rede aleatória com $p \approx 1$, e também permitindo uma rede com topologia intermediária $0 < p < 1$ \cite{lopes2011redes}.
 \begin{figure}[!htb]
 	\caption{Variação do $p$ sem alterar o numero de vértices $v = 20$ e $k = 4$.}
 	\begin{center}
 		\includegraphics[width=0.9\linewidth]{imagens/watts-sm}
 	\end{center}
 	\small{Fonte: \citet{watts1998collective}.}
 	\label{fig:watts-sm}
 \end{figure} 
 
 As redes \textit{small-worlds} são caracterizadas com duas principais medidas: O tamanho do caminho $L(p)$ e o coeficiente de agrupamento ou coeficiente de aglomeração $C(p)$. $L(p)$ é medido como a média do caminho mais curto de todos pares de vértices \cite{lopes2011redes}. O coeficiente de aglomeração será melhor definido na seção \ref{subsec:coeficienteaglomeracao}.
 
 \subsection{Redes Livres de Escala}
 As redes livres de escala, foram propostas por \citet{barabasi1999emergence}, nelas um nó tem a probabilidade \(P(k)\) de possuir \(k\) arestas obedecendo a lei da potência \(P(k) \sim k^{-\gamma}\), onde $\gamma$ determina o decaimento exponencial \cite{Antiqueira2005, lopes2011redes}. Segundo \citet{Viana2007} nas redes livres de escala muitos nós tem poucas arestas e poucos nós se ligam a muitos.
 
A rede livre de escala é baseado em duas regras: crescimento e preferência linear de ligação. A redes é gerada com a inclusão de $n_0 < n$ vértices aleatoriamente conectados sendo estas inserções antes do crescimento da rede. Na etapa de crescimento da rede, a cada instante de tempo $t$ um novo vértice $v$ é adicionado na rede, o número de arestas desse vértice segue uma preferencia linear de ligação da fórmula:
\[ P(v_i \leftrightarrow v_j) = \frac{k_j}{\sum_u k_U}, \forall v_u \in V \]

Sendo $P$ a probabilidade de um vértice $v_i$ ser conectado ao novo vértice $v_j$ é linearmente proporcional ao grau $k_j$ do vértice $v_j$  \citet{lopes2011redes}.

 \begin{figure}[!htb]
	\caption{Lei de potência do número de conexões k. De cima pra baixo, as curvas de lei de potência foram obtidas com o parâmetro \(\gamma\) definido como: 0.5 , 1 , 1.5 , 2, 2.5, 3, 4 e 5.}
	\begin{center}
		\includegraphics[width=0.95\linewidth]{imagens/scalefree}
	\end{center}
	\small{Fonte: \citet{lopes2011redes}.}
	\label{fig:lei-potencia}
\end{figure} 
 
É possível ver na Figura \ref{fig:lei-potencia} que quanto maior o \(\gamma\) menos existem vértices com maior número de arestas. O paradigma de ligação dos novos vértices também é conhecido como "o rico fica mais rico" \cite{costa}.
     
\subsection{Redes Aleatórias}
As redes aleatórias, foram inicialmente propostas em 1959 e podem ser consideradas o modelo mais elementar de rede complexas \cite{lopes2011redes, erdds1959random}.
Segundo \citet{Viana2007} e \citet{lopes2011redes} as redes aleatórias são um sistema formado por $E$ arestas e $N$ vértices, inicialmente são dispostos os vértices, e depois aleatoriamente as arestas são distribuídas com probabilidade igual entre os vértices, evitando auto-relacionamentos e conexões múltiplas entre dois vértices.


 
\subsection{Coeficiente de Aglomeração}
\label{subsec:coeficienteaglomeracao}
Segundo \citet{Viana2007} o coeficiente de aglomeração mede o quão conectado estão os nós da rede ou do grafo. \citet{Antiqueira2005}  define o coeficiente de aglomeração sendo:\[CA_i = \frac{E_i}{k_i(k_i-1)}\]

\citet{Antiqueira2005}  continua: “Sendo para cada vértice \(i\) existe \(k_i\) arestas, que os ligam a outros \(k_i\) vértices. Se esses \(k_i\) vértices estivessem ligados diretamente à todos os outros vértices do conjunto, haveriam \(k_i(k_i- 1)\) arestas entre eles. E assumindo \(E_i\) o número de arestas que existentes entre os \(k_i\) vértices.	O coeficiente da rede inteira é a média de todos \(CA_i\)”.

\begin{figure}[!h] \centering
	\centering
	\caption{Coeficiente de aglomeração.}
	\subfloat[]{
		\begin{tikzpicture}
		\begin{scope}[xshift=4cm]
		\node[targetnode] (1) {$1$};
		\node[main node] (2) [above right = 2cm  of 1] {$2$};
		\node[main node] (3) [below right = 2cm  of 1] {$3$};
		\node[main node] (4) [right = 2cm  of 2] {$4$};
		\node[main node] (5) [right = 2cm  of 3] {$5$};
		
		\path[draw,thick]
		(1) edge node {} (2)
		(1) edge node {} (3)
		(1) edge node {} (4)
		(1) edge node {} (5)	
		(2) edge node {} (3)
		(2) edge node {} (4)
		(2) edge node {} (5)
		(3) edge node {} (4)
		(3) edge node {} (5)
		(4) edge node {} (5)
		;
		\end{scope}
		\end{tikzpicture}
		\label{fig:grafo-aglomeracao1}
	}
	\subfloat[]{
		\begin{tikzpicture}
		\begin{scope}[xshift=4cm]
		\node[targetnode] (1) {$1$};
		\node[main node] (2) [above right = 2cm  of 1] {$2$};
		\node[main node] (3) [below right = 2cm  of 1] {$3$};
		\node[main node] (4) [right = 2cm  of 2] {$4$};
		\node[main node] (5) [right = 2cm  of 3] {$5$};
		
		\path[draw,thick]
		(1) edge node {} (2)
		(1) edge node {} (3)
		(1) edge node {} (4)
		(1) edge node {} (5)
		(2) edge node {} (4)
		(3) edge node {} (5)
		(4) edge node {} (5)
		;
		\end{scope}
		\end{tikzpicture}
		\label{fig:grafo-aglomeracao2}
	}
	\
	\small{\leftline{Fonte: Autor.}}
	\label{fig:grafo-aglomeracao}
\end{figure}

Na Figura \ref{fig:grafo-aglomeracao1}, assumindo o peso de todas as arestas como 1, o coeficiente de aglomeração do vértice em azul é 1, já na Figura \ref{fig:grafo-aglomeracao2} o coeficiente do vértice em azul segundo a fórmula é $\frac{1}{2}$. 
    


\section{\textit{League of Legends}}
\label{chap:lol}
O jogo \textit{League of Legends} é um jogo classificado como arena de batalha online de multi jogadores (do inglês \textit{Multiplayer Online Battle Arena} ) ou conhecido também como MOBA, que é um estilo de jogo onde duas equipes se enfrentam em um campo de batalha e cada jogador controla o seu personagem, mais chamado de herói ou campeão. O objetivo do MOBA é derrotar a equipe adversária destruindo a construção principal da equipe inimiga.

	A arena onde acontece o jogo é uma arena onde normalmente o mapa inicialmente é  espelhado,  ou  seja, o lado que cada time está não  oferece vantagens  exclusivas.  O mapa é composto  de  três caminhos até a base inimiga.
    
%%preciso conferir se essa foto é acervo, ou é do lol

\begin{figure}[H]
	\caption{Mapa do jogo de \textit{League of Legends}}
	\begin{center}
		\includegraphics{imagens/mapa_lol.jpg}
	\end{center}
	\small{Fonte: Autor (2018).}
	\label{fig:mapa_lol}
\end{figure}

	A Figura \ref{fig:mapa_lol} mostra como é realmente o mapa do jogo, sendo que os círculos mostram a localização da construção principal, e os triângulos as construções de suporte de cada equipe, sendo azul uma equipe e vermelho a outra.
    
	Com o início do jogo cada jogador escolhe  um  herói  diferente, onde  cada  herói tem um conjunto de características únicas, como habilidades especiais,  seu impacto  no jogo,  na equipe adversária e na equipe aliada.
    
	Depois de escolher os heróis de cada equipe, cada jogador deve procurar adquirir recursos no jogo e objetivos para conseguir vantagens. Os recursos são limitados por equipe e por tempo, ou seja, deve ser bem escolhido quem ficará com a maior parte dos recursos da equipe.

\section{D3.js}
A biblioteca D3.js é uma biblioteca em JavaScript especializada em dar vida aos dados. Segundo \citet[tradução do autor]{d3cook} "Em certo sentido, o D3 é uma biblioteca JavaScript especializada que permite criar incríveis visualizações de dados usando uma abordagem mais simples (baseada em dados), aproveitando os padrões da \textit{Web} existentes", e a organização oficial continua :


\begin{quote}
	O D3.js é uma biblioteca em JavaScript para manipulação de documentos por dados. D3 te ajuda a trazer vida para os dados utilizando HTML, SVG e CSS. A ênfase da biblioteca D3 nos padrões da \textit{web} dá-lhe todos os recursos dos navegadores modernos sem te prender a um \textit{framework} proprietário, combinando poderosos componentes de visualização e uma aproximação orientada por dados da manipulação do DOM.
	\cite[tradução do autor]{d3js}
\end{quote}


Com essa biblioteca é possível dar vida à informações e dados, como por exemplo a Figura \ref{fig:d3} que possui duas das diversas maneiras de se exibir informações com o D3.js . E com o D3 foi gerado o grafo do \textit{web service}, programa este que será clarificado na seção seguinte.

\begin{figure}[H]
	\caption{Exemplo de visualização de dados utilizando o D3.js.}
	\begin{center}
		\includegraphics[width=7.5cm]{imagens/d3_1.png}%
		\hspace{1cm}% add some horizontal spacing
		\includegraphics[width=7.5cm]{imagens/d3_2.PNG}%
	\end{center}
	\small{Fonte: \cite{d3js}.}
	\label{fig:d3}
\end{figure}

\section{Estado da Arte}


{\Huge VOU REFAZER}
