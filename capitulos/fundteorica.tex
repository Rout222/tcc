\chapter{Fundamentação Teórica}
\label{chap:fundteorica}
Nessa parte será apresentados conceitos para melhor entendimento deste trabalho e da modelação utilizada. 
A seção \ref{chap:grafos} elucida o que são grafos e o capítulo \ref{chap:redecomplexa} explica a variação de grafo utilizado para modelar o problema proposto. Depois na seção \ref{chap:lol} é esclarecido o ambiente que será abstraído, explicando algumas regras básicas do jogo.
\section{Grafos}
\label{chap:grafos}

Quando descrevemos uma situação utilizando pontos e ligações entre algum desses pontos, como o exemplo os pontos sendo pessoas e as ligações entre as pessoas sendo as amizades feitas, a abstração matemática desse tipo dá lugar ao conceito de grafo \cite{Lucchesi1979}.

Segundo \citet{Viana2007} "Um grafo \(G(V, E)\) pode ser definido como um conjunto de vértices \(V\), e um conjunto de conexões \(E\) " e ele continua: "Cada elemento do conjunto \(E\), associa dois elementos do conjunto \(V\), assim, se \((u,v) \in E\), então existe uma conexão entre os vértices u e o vértice v .". Sendo eles classificados como vizinhos ou adjacente \cite{grafosucinto}.


\subsection{Binário e Não-binário}
Os grafos também podem ser classificados como binários e não-binários, sendo os binários suas arestas representadas por 1 ( se existem ) ou 0 ( se não existem ) e os não-binários quando existe um conjunto \(W\) onde informa a intensidade da interação de uma aresta entre dois vértices  \cite{Viana2007}.
Na Figura \ref{fig:grafo-exemplobinario} temos um grafo binário onde se \((u,v) \in E\) então ela está representada no grafo, já na Figura \ref{fig:grafo-exemplonaobinario} além da existência da conexão entre os vértices, está representado o conjunto \(W\) informando o peso ou aresta daquela aresta.

\begin{figure}[!h] \centering
	\centering
	\caption{Exemplo de grafos.}
	\subfloat[Grafo binário com quatro vértices e quatro arestas.]{
		\begin{tikzpicture}
		\begin{scope}[xshift=4cm]
		\node[main node] (1) {$1$};
		\node[main node] (2) [right = 2cm  of 1] {$2$};
		\node[main node] (3) [below = 2cm  of 1] {$3$};
		\node[main node] (4) [right = 2cm  of 3] {$4$};
		
		\path[draw,thick]
		(1) edge node {} (2)
		(1) edge node {} (4)
		(4) edge node {} (2)
		(4) edge node {} (3)
		;
		\end{scope}
		\end{tikzpicture}
		\label{fig:grafo-exemplobinario}
	}
	\subfloat[Grafo não-binário com quatro vértices e três arestas.]{
		\begin{tikzpicture}
		\begin{scope}[xshift=4cm]
		\node[main node] (1) {$1$};
		\node[main node] (2) [right = 2cm  of 1] {$2$};
		\node[main node] (3) [below = 2cm  of 1] {$3$};
		\node[main node] (4) [right = 2cm  of 3] {$4$};
		
		\path[draw,thick]
		(1) edge node {2} (2)
		(1) edge node {1} (3)
		(2) edge node {1} (4)
		;
		\end{scope}
		\end{tikzpicture}
		\label{fig:grafo-exemplonaobinario}
	}
	\
	\small{\leftline{Fonte: Autor.}}
	\label{fig:grafo-exemplo}
\end{figure}


\subsection{Vizinhança}
\citet{grafosucinto} diz sobre a vizinhança de um grafo que "O conjunto de vértices \(X\) de um grafo \(G\) é o conjunto de todos os vértices que tem algum vizinho em \(X\)", e diz que esse conjunto de vértices pode ser chamado de \(\Gamma_G(X)\). Exemplificando, o conjunto \(\Gamma_G(1)\) da Figura \ref{fig:grafo-exemplobinario} são os vértices \(2\) e \(4\) e a aresta \( (2,4) \) sendo representados na Figura \ref{fig:grafo-vizinhanca}.

\begin{figure}[H] \centering
	\centering
	\caption{Exemplo de vizinhança de um vértice de um grafo.}
		\begin{tikzpicture}
		\begin{scope}[xshift=4cm]
		\node[targetnode] (1) {$1$};
		\node[main node] (2) [right = 2cm  of 1] {$2$};
		\node[offnode] (3) [below = 2cm  of 1] {$3$};
		\node[main node] (4) [right = 2cm  of 3] {$4$};

		\path[draw, style={offline}]
		(1) edge node {} (2)
		(1) edge node {} (4)
		(3) edge node {} (4)
		;
		\path[draw,thick]
		(4) edge node {} (2)
		;
		\end{scope}
		\end{tikzpicture}
	\small{\leftline{Fonte: Autor.}}
	\label{fig:grafo-vizinhanca}
\end{figure}

\subsection{Grau}

O grau de um vértice \(v\) é definido por \citet{grafosucinto} sendo a quantidade de arestas que chegam no vértice \(v\), sendo denotado por \(g(v)\). Continuando ainda no exemplo da Figura \ref{fig:grafo-exemplobinario} o \(g(1)\) é 2 pois apenas as arestas \((1,2)\) e \((1,4)\) incidem no vértice 1.

O grau máximo de um grafo \(G\) é o número do vértice que tem o maior grau presente nesse grafo, ou seja \(\Delta(G) = max\{g(v) : v \in V(G)\}\). E o grau mínimo de um grafo \(G\) é o grau do vértice com menor grau, \(\delta(G) = min\{g(v) : v \in V(G)\}\). E ainda, um grafo é regular se \(\delta(G) = \Delta(G)\) e é \(k\)-regular se \(\delta(G) = \Delta(G) = k\) \cite{grafosucinto}.


E quando modelamos um sistema real em forma de grafos vamos ver na seção seguinte que esse grafo recebe um nome especial, sendo chamado de rede complexa.

\section{Redes Complexas}
\label{chap:redecomplexa}
Sobre redes complexas \citet{Viana2007} diz que é utilizado o termo redes complexas quando um grafo representa um sistema físico real, então levando em conta isso, um grafo do jogo LOL pode ser considerado como uma rede complexa.

Para que seja possível classificar os resultados adequadamente, serão apresentados três modelos que se destacam no estado da arte segundo \citet{Albert2002}: As redes \textit{small worlds}, as redes livres de escala e as redes aleatórias. E também será explicado sobre o coeficiente de aglomeração .

\subsection{\textit{Redes Small Worlds}}
 As redes \textit{small worlds} são redes que o caminho entre dois nós são relativamente pequenos. (continua...).
 
 
 \subsection{Redes Livres de Escala}
 Nas redes livres de escala um nó tem a probabilidade \(P(k)\) de possuir \(k\) arestas obedecendo a lei da potência \(P(k) \sim k^{-y}\) \cite{Albert2002,Antiqueira2005}. Segundo \citet{Viana2007} nas redes livres de escala muitos nós tem poucas arestas e poucos nós se ligam a muitos.
 
     
\subsection{Redes Aleatórias}
Segundo \citet{Viana2007} as redes aleatórias são um sistema formado por E arestas e N vértices, onde as arestas são distribuídas aleatoriamente. \citet{CunhaRecuero2004} apud ( tenho que ver certinho ) exemplifica esse tipo de rede como uma festa, onde “bastava uma conexão entre cada um dos convidados de uma festa, para que todos estivessem conectados ao final dela” e que quanto mais conexões forem criadas, maior a chance de serem criados grupos de pessoas que de tempos em tempos se relacionavam com outros grupos e que poderiam concluir que esses nós se relacionavam de forma randômica.

 
 \subsection{Coeficiente de Aglomeração}
 Segundo \citet{Viana2007} o coeficiente de aglomeração mede o quão conectado estão os nós da rede ou do grafo. \citet{Antiqueira2005}  define o coeficiente de aglomeração sendo:\[CA_i = \frac{E_i}{k_i(k_i-1)}\]


\citet{Antiqueira2005}  continua: “Sendo para cada vértice \(i\) existe \(k_i\) arestas, que os ligam a outros \(k_i\) vértices. Se esses \(k_i\) vértices estivessem ligados diretamente à todos os outros vértices do conjunto, haveriam \(k_i(k_i- 1)\) arestas entre eles. E assumindo \(E_i\) o número de arestas que existentes entre os \(k_i\) vértices.	O coeficiente da rede inteira é a média de todos \(CA_i\)”.

\begin{figure}[!h] \centering
	\centering
	\caption{Coeficiente de aglomeração.}
	\subfloat[]{
		\begin{tikzpicture}
		\begin{scope}[xshift=4cm]
		\node[targetnode] (1) {$1$};
		\node[main node] (2) [above right = 2cm  of 1] {$2$};
		\node[main node] (3) [below right = 2cm  of 1] {$3$};
		\node[main node] (4) [right = 2cm  of 2] {$4$};
		\node[main node] (5) [right = 2cm  of 3] {$5$};
		
		\path[draw,thick]
		(1) edge node {} (2)
		(1) edge node {} (3)
		(1) edge node {} (4)
		(1) edge node {} (5)	
		(2) edge node {} (3)
		(2) edge node {} (4)
		(2) edge node {} (5)
		(3) edge node {} (4)
		(3) edge node {} (5)
		(4) edge node {} (5)
		;
		\end{scope}
		\end{tikzpicture}
		\label{fig:grafo-aglomeracao1}
	}
	\subfloat[]{
		\begin{tikzpicture}
		\begin{scope}[xshift=4cm]
		\node[targetnode] (1) {$1$};
		\node[main node] (2) [above right = 2cm  of 1] {$2$};
		\node[main node] (3) [below right = 2cm  of 1] {$3$};
		\node[main node] (4) [right = 2cm  of 2] {$4$};
		\node[main node] (5) [right = 2cm  of 3] {$5$};
		
		\path[draw,thick]
		(1) edge node {} (2)
		(1) edge node {} (3)
		(1) edge node {} (4)
		(1) edge node {} (5)
		(2) edge node {} (4)
		(3) edge node {} (5)
		(4) edge node {} (5)
		;
		\end{scope}
		\end{tikzpicture}
		\label{fig:grafo-aglomeracao2}
	}
	\
	\small{\leftline{Fonte: Autor.}}
	\label{fig:grafo-aglomeracao}
\end{figure}

Na Figura \ref{fig:grafo-aglomeracao1}, assumindo o peso de todas as arestas como 1, o coeficiente de aglomeração do vértice em azul é 1, já na Figura \ref{fig:grafo-aglomeracao2} o coeficiente do vértice em azul segundo a fórmula é $\frac{1}{2}$. 
    


\section{\textit{League of Legends}}
\label{chap:lol}
O jogo \textit{League of Legends} é um jogo classificado como arena de batalha online de multi jogadores (do inglês \textit{Multiplayer Online Battle Arena} ) ou conhecido também como MOBA, que é um estilo de jogo onde duas equipes se enfrentam em um campo de batalha e cada jogador controla o seu personagem, mais chamado de herói ou campeão. O objetivo do MOBA é derrotar a equipe adversária destruindo a construção principal da equipe inimiga.

	A arena onde acontece o jogo é uma arena onde normalmente o mapa inicialmente é  espelhado,  ou  seja, o lado que cada time está não  oferece vantagens  exclusivas.  O mapa é composto  de  três caminhos até a base inimiga.
    
%%preciso conferir se essa foto é acervo, ou é do lol

\begin{figure}[H]
	\caption{Mapa do jogo de \textit{League of Legends}}
	\begin{center}
		\includegraphics{imagens/mapa_lol.jpg}
	\end{center}
	\small{Fonte: Autor (2018).}
	\label{fig:mapa_lol}
\end{figure}

	A Figura \ref{fig:mapa_lol} mostra como é realmente o mapa do jogo, sendo que os círculos mostram a localização da construção principal, e os triângulos as construções de suporte de cada equipe, sendo azul uma equipe e vermelho a outra.
    
	Com o início do jogo cada jogador escolhe  um  herói  diferente, onde  cada  herói tem um conjunto de características únicas, como habilidades especiais,  seu impacto  no jogo,  na equipe adversária e na equipe aliada.
    
	Depois de escolher os heróis de cada equipe, cada jogador deve procurar adquirir recursos no jogo e objetivos para conseguir vantagens. Os recursos são limitados por equipe e por tempo, ou seja, deve ser bem escolhido quem ficará com a maior parte dos recursos da equipe.

\section{Estado da Arte}


{\Huge VOU REFAZER}


No estado da arte, estudos com jogos de gênero MOBA são poucos, mas relacionados a redes complexas e predição existem diversos.

\subsection{Identifying Patterns in Combat that are Predictive of success}
 O trabalho de Yang \textit{et al}, relacionou a posição dos jogadores em uma batalha do MOBA com redes complexas, ao predizer o time vencedor daquela batalha.

\begin{figure}[!ht]
	\caption{As estruturas gráficas de uma batalha de MOBA}
	\begin{center}
		\includegraphics[width=15cm]{imagens/yang.PNG}
	\end{center}
	\small{Fonte: \cite{Yang2014}.}
	\label{fig:yang2014}
\end{figure}

Yang \textit{et al} modela os dados de um combate, em redes complexas, e treina uma arvore de decisão usando as melhores características tiradas dessa rede complexa. Depois de treinado a árvore, é identificado regras de combate que são preditivas de sucesso, e depois essas regras são traduzidas de volta em um padrão de combate específico usando uma técnica chamada mineração de subgrafo frequente.

\subsection{Análise de uma Métrica Alternativa para Predição de Laços Sociais em Grafos Lei de Potência}

Feito por \citeauthor{Danielewicz2016}, ela mostra uma visão geral do problema de predição de laços sociais, analisa os modelos de geração de grafos principalmente que seguem a lei da potência, no âmbito de formação de laços sociais e depois consegue uma métrica para predição de laços. Ela faz testes usando técnicas diferentes, mostrando seus resultados.